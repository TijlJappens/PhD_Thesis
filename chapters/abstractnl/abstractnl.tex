% !TeX root = ../../thesis.tex
\chapter{Beknopte samenvatting}

Deze doctoraatsthesis presenteert een uitgebreide studie van symmetriebeschermde topologische (SPT) fasen in kwantumspinsystemen, met de nadruk op hun intrigerende eigenschappen onder verschillende aanvullende symmetrieën. Het werk is verdeeld in twee hoofdartikelen, die elk unieke inzichten bijdragen aan het vakgebied van de gecondenseerde materie fysica.
\\\\
De scriptie begint met een op zichzelf staande inleiding in SPT-orde, die dient als een fundamenteel begrip voor latere besprekingen. Startend vanuit kwantumcircuits met eindige diepte, biedt de inleiding essentiële achtergrondkennis en legt zo de basis voor het verkennen van topologisch gedrag in kwantumspinsystemen met symmetrieën.
\\\\
Het eerste artikel richt zich op de classificatie van $G$-lading Thouless-pompen binnen ééndimensionale inverteerbare toestanden. Deze pompen beschrijven lussen van symmetriebeschermde topologische (SPT) kwantumtoestanden, gekarakteriseerd door continue transformaties van systeemparameters die het systeem terugbrengen naar zijn oorspronkelijke toestand na het voltooien van een volledige lus. Het artikel analyseert systematisch deze lussen voor verschillende symmetriegroepen, wat leidt tot een volledig begrip van de classificatie van lussen van eendimensionale bosonische SPT's. Opmerkelijk genoeg generaliseert dit werk het concept van Thouless-pompen, omvattende gevallen waarin de symmetriegroep $U(1)$ is en wordt geclassificeerd door een geheel getal dat de ladings eenheid vertegenwoordigt die per cyclus wordt gepompt.
\\\\
Het tweede artikel onderzoekt tweedimensionale kwantumspinsystemen die SPT-fasen bevatten met een symmetrie gelocaliseerd in een punt beschreven door een eindige groep $G$. Door het opleggen van translatie-invariantie in verschillende richtingen, ontdekt het artikel nieuwe topologische invarianten. Specifiek wordt er een nieuwe index ontdekt die waarden aanneemt in $H^2(G,\TT)$, naast de eerder bekende index die waarden aanneemt in $H^3(G,\TT)$. Bovendien, wanneer translatie-invariantie in twee richtingen wordt opgelegd, ontstaat er een extra index die waarden aanneemt in $H^1(G,\TT)$. De belangrijkste bijdrage hier is de wiskundige nauwkeurigheid in de constructie en het bewijs van de stabiliteit van deze indices.
\\\\
Beide artikelen dragen aanzienlijk bij aan het begrip van topologisch gedrag in kwantumspinsystemen met symmetrieën, waarbij de kracht van wiskundige nauwkeurigheid wordt getoond bij het construeren en begrijpen van topologische invarianten. De combinatie van grondige analyse en toegankelijke inleidingen zorgt ervoor dat de scriptie een waardevolle bron is voor onderzoekers en studenten die de fascinerende wereld van SPT-fasen in de gecondenseerde materie fysica verkennen.

%%%%%%%%%%%%%%%%%%%%%%%%%%%%%%%%%%%%%%%%%%%%%%%%%%
% Keep the following \cleardoublepage at the end of this file, 
% otherwise \includeonly includes empty pages.
\cleardoublepage

% vim: tw=70 nocindent expandtab foldmethod=marker foldmarker={{{}{,}{}}}
