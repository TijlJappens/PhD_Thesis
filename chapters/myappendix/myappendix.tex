% !TeX root = ../../thesis.tex
\chapter{Appendix to chapter \ref{ch:TranslationSPT}}\label{ch:myappendix}

\section{Properties of locally generated automorphisms: 1d}\label{sec:properties-of-locally-generated-automorphisms-1d}
In this section, let $\AA$ be a quasi-local $C^*$ algebra over $\ZZ$.
\begin{lemma}\label{lem:PropertiesLocallyGeneratedAutomorphisms1d}
	Take $\Phi\in\BB_{F_\phi}([0,1])$ (with $F_\phi$ as presented in \eqref{eq:OurFFunction}) then there exists an $A\in\UU(\AA)$ satisfying
	\begin{equation}
		\Ad{A}=\gamma^{\Phi_L+\Phi_R}_{0;1}\circ\gamma^\Phi_{1;0}.
	\end{equation}
	Moreover, this $A$ is summable (this was defined in subsection \ref{sec:QuasiLocalC*Algebra}). More generally there exists a continuous (in norm topology) one-parameter family $\lambda\in[0,1]\mapsto A_\lambda\in\UU(\AA)$ such that $A_\lambda$ is summable for each $\lambda$ and such that
	\begin{equation}
		\Ad{A_\lambda}=\gamma^{\Phi_L+\Phi_R}_{0;\lambda}\circ\gamma^\Phi_{\lambda;0}.
	\end{equation}
\end{lemma}
\begin{proof}
	We have for any $B\in\AA$ that
	\begin{align}
		\gamma^{\Phi_L+\Phi_R}_{0;t}\circ\gamma^\Phi_{t;0}(B)&=B+\int_{0}^{t}\d s\frac{\d}{\d s}\left(\gamma^{\Phi_L+\Phi_R}_{0;s}\circ\gamma^\Phi_{s;0}(B)\right)\\
		&=B-i\int_{0}^{t}\d s\sum_{I\in\mathfrak{G}_\ZZ}\gamma^{\Phi_L+\Phi_R}_{0;s}([\Phi_L(s,I)+\Phi_R(s,I)-\Phi(s,I),\gamma^{\Phi}_{s;0}(B)])\\
		\label{eq:EquationShowingBounded_H}
		&=B-i\int_{0}^{t}\d s\gamma^{\Phi_L+\Phi_R}_{0;s}\circ\gamma^{\Phi}_{s;0}\left(\left[\sum_{I\in\mathfrak{G}_\ZZ}\gamma^{\Phi}_{0;s}(\Phi_L(s,I)+\Phi_R(s,I)-\Phi(s,I)),B\right]\right).
	\end{align}
	To simplify notation we will define an interaction
	\begin{equation}
		\tilde\Phi(s,I)=\Phi_L(s,I)+\Phi_R(s,I)-\Phi(s,I).
	\end{equation}
	We will in essence have to find a 1D analogue of the proof of Theorem 5.2 of \cite{Ogata2d} with a few exceptions. Most notably we need the summability condition. First, let us still define the extensions
	\begin{equation}
		X(m)\defeq \{x\in\ZZ|\exists y\in X:\textrm{dist}(x,y)\leq m\}
	\end{equation}
	and
	\begin{equation}
		\Delta_{X(m)}\defeq \Pi_{X(m)}-\Pi_{X(m-1)}
	\end{equation}
	the differences between the conditional expectation values. Define an interaction
	\begin{equation}\label{eq:RecastingTheInteraction}
		\Xi(Z,t)\defeq\sum_{m\geq 0}\sum_{\substack{X\subset Z:\\X(m)=Z}}\Delta_{X(m)}(\gamma^{\Phi}_{0;t}(\tilde{\Phi}(X,t)))
	\end{equation}
	then we get by construction that
	\begin{equation}
		\gamma^{\Phi_L+\Phi_R}_{0;t}\circ\gamma^{\Phi}_{t;0}=\gamma^{\Xi}_{0;t}.
	\end{equation}
	We will now define
	\begin{equation}
		A_\lambda\defeq \mathcal{T}\exp(-i\int_0^\lambda \d s\: \sum_{Z\in\mathfrak{B}_{\ZZ}}\Xi(Z,s)).
	\end{equation}
	By construction, if this exists it is continuous in $\lambda$. To show that it exists and is summable, define some sets
	\begin{align}
		B_n&\defeq [-n,n]\cap\ZZ&B_{n,L}&\defeq ]-\infty,-n]\cap\ZZ&B_{n,R}&\defeq[n,\infty[\cap\ZZ
	\end{align}
	and let
	\begin{equation}
		A_n\defeq \mathcal{T}\exp(-i\int_0^1 \d s\: \sum_{Z\in\mathfrak{B}_{\ZZ}}\Xi_{B_n}(Z,s)).
	\end{equation}
	In what follows we will find a bound on
	\begin{equation}
		c_n\defeq \sup_{t\in[0,1]}\norm{\sum_{Z\in\mathfrak{B}_\ZZ}(\Xi(Z,t)-\Xi_{B_n}(Z,t))}
	\end{equation}
	that is summable. Finding this would conclude the proof because
	\begin{equation}
		\sum_n\norm{A-A_n}=\sum_n\norm{\int_0^1\d s\: \frac{\d}{\d s}A_{n}(s)^\dagger A(s)}\leq \sum_n \sup_{t\in[0,1]}\norm{\sum_{Z\in\mathfrak{B}_\ZZ}(\Xi(Z,t)-\Xi_{B_n}(Z,t))}=\sum_n c_n.
	\end{equation}
	In analogy with equation (5.22) of \cite{Ogata2d} we get
	\begin{align}
		\label{eq:c_n_Bound1D}
		c_n&\leq \sum_{\substack{Z\in\mathfrak{B}_{\ZZ}:\\Z\cap B_{n,L}\neq\emptyset\\\text{or}\\Z\cap B_{n,R}\neq\emptyset}}\sup_{t\in[0,1]}\norm{\Xi(Z,t)}\\
		&\leq \frac{8}{c_F}(e^{2I_F(\Phi)}-1) \sum_{m\geq 0}\sum_{\substack{X\in\mathfrak{B}_{\ZZ}:\\X(m)\cap B_{n,L}\neq\emptyset\\\text{or}\\X(m)\cap B_{n,R}\neq\emptyset}}\sup_{t\in[0,1]}\left(\norm{\tilde{\Phi}(X,t)}\right)\abs{X}G_F(m).
	\end{align}
	Now using a trick analogous to what was done in equation (5.27) of \cite{Ogata2d} we get
	\begin{equation}
		c_n\leq \norm{\Phi_1}_F\left(\sum_{m\geq 0}G_F(m)\right)(\sum_{x\in B_{n,L}}\sum_{y=B_{0,R}}F(d(x,y))+\sum_{x\in B_{0,L}}\sum_{y=B_{n,R}}F(d(x,y))).
	\end{equation}
	To show that the last bound is summable we only have to observe that
	\begin{equation}
		\sum_{n=0}^\infty\sum_{x\in B_{n,L}}\sum_{y=B_{0,R}}F(d(x,y))\leq \sum_{x\in C_1}\sum_{y=C_2}F(d(x,y)).
	\end{equation}
	where $C_1$ is the cone one obtains by putting the $B_{n,L}$ on top of each other whereas $C_2$ is the right half plane of $\ZZ^2$. The proof that the sum over the cones is bounded is done explicitly in \cite{Ogata2d}.
\end{proof}
\section{Properties of locally generated automorphisms: 2d}\label{sec:properties-of-locally-generated-automorphisms-2d}
In this section, let $\AA$ be a quasi-local $C^*$ algebra over $\ZZ^2$. Take $H\in\BB_{F_\phi}([0,1])$ (for some $0<\phi<1$) and $\gamma^H_{s;t}$ its locally generated automorphism (see section \ref{sec:Interactions}). We will now show some properties of this locally generated automorphism $\gamma^H_{s;t}$. In this section we will heavily rely on the framework and theorems of Yoshiko Ogata \cite{Ogata2d} who in her turn relies heavily on the framework developed in \cite{doi:10.1063/1.5095769}. We now need two additional classes of automorphisms:
\begin{definition}
	Take $\alpha\in\Aut{\AA}$. We say that $\alpha\in\textrm{HAut}_1(\AA)$ if and only if for any $0<\theta<\pi/2$ there exists an $a\in\UU(\AA)$ and some $\alpha_\sigma\in\Aut{\AA_{\nu^{\sigma}(C_\theta\cap\sigma)}}$ for each $\sigma\in\{L,R\}$ such that
	\begin{equation}
		\alpha=\Ad{a}\circ\alpha_L\otimes\alpha_R.
	\end{equation}
\end{definition}
\begin{definition}
	Take $\alpha\in\Aut{\AA}$. We say that $\alpha\in\textrm{VAut}_1(\AA)$ if and only if there exists some $a\in\AA$, $\alpha_U\in\Aut{\tau(\AA_{C_\theta^c}\cap\AA_{U})}$ and an $\alpha_D\in\Aut{\tau^{-1}(\AA_{C_\theta^c}\cap\AA_{D})}$ such that
	\begin{equation}
		\alpha=\Ad{a}\circ\alpha_U\otimes\alpha_D.
	\end{equation}
	If furthermore $\alpha_U\circ\beta_g^U=\beta_g^U\circ\alpha_U$ we say that $\alpha\in\textrm{GVAut}_1(\AA)$.
\end{definition}
Additionally, remember the definition of the $\textrm{QAut}_2(\AA)$ that was presented in definition \ref{def:SQAut2}. Now we will state some properties of locally generated automorphisms:
\begin{lemma}\label{lem:PropertiesLocallyGeneratedAutomorphisms}
	Take $H$ an interaction such that there exists a $0<\phi<1$ satisfying that $\norm{H}_{f_\phi}\leq 1$. The following statements now hold (for any $s,t\in\RR$):
	\begin{enumerate}
		\item $\gamma^H_{s;t}\circ\gamma^{H_D}_{t;s}\otimes\gamma^{H_U}_{t;s}\in\textrm{HAut}_1(\AA)$.
		\item $\gamma^H_{s;t}\circ\gamma^{H_L}_{t;s}\otimes\gamma^{H_R}_{t;s}\in\textrm{VAut}_1(\AA)$. If additionally $H$ is a $G$-invariant interaction we even have $\gamma^H_{s;t}\circ\gamma^{H_L}_{t;s}\otimes\gamma^{H_R}_{t;s}\in\textrm{GVQAut}_1(\AA)$. The similar statement also holds if we replace $L$ and $R$ by $\nu^{-1}(L)$ and $\nu(R)$.
		\item $\gamma^{H_U}_{s;t}\otimes\gamma^{H_D}_{s;t}\in\textrm{SQAut}_1(\AA)$. If additionally $H$ is a $G$-invariant interaction we even have $\gamma^{H_U}_{s;t}\otimes\gamma^{H_D}_{s;t}\in\textrm{GSQAut}_1(\AA)$.
		\item If $H$ is $G$-invariant then $\gamma^{H}_{t;s}\circ\beta_g^U\circ\gamma^{H}_{s;t}\circ(\beta_g^U)^{-1}\in\textrm{HAut}_1(\AA)$.
		\item $\gamma^{H}_{t;s}\in\textrm{SQAut}_1(\AA)$.
	\end{enumerate}
\end{lemma}
\begin{proof}
	Part 1 is done in Proposition 5.5 in \cite{Ogata2d} and Part 4 follows trivially from Part 1. Except for the translation operators in my definition of the $\textrm{SQAut}(\AA)$, part 3 follows from Theorem 5.2 in \cite{Ogata2d}. To show part 3 we therefore have to show that Theorem 5.2 in \cite{Ogata2d} still holds if we replace $\mathcal{C}_0$ and $\mathcal{C}_1$ in the proof by the sets highlighted in figure \ref{fig:SetupSQAut2}. Take the $\Psi$ from this proof to be our $H$, take $\Psi^{(0)}$ to be $\sum_{C\in\mathcal{C}_0}H_{C}$ (with our new definition of $\mathcal{C}_0$) and take $\Psi^{(1)}\defeq \Psi-\Psi^{(0)}$. Now define $\Xi^{(s)}(Z,t)$ through
	\begin{equation}\label{eq:PropertiesLocallyGeneratedAutomorphismsProofDefinitionXi}
		\Xi^{(s)}(Z,t)\defeq\sum_{m=0}^\infty \sum_{\substack{X\subset Z,\\X(m)=Z}}\Delta_{X(m)}(\gamma^\Psi_{s;t}(\Psi^{(1)}(X;t)))
	\end{equation}
	with these new definitions of $\Psi^{(0)}$ and $\Psi^{(1)}$. We now want to show that for every $t$,
	\begin{equation}
		\sum_{Z\subset\ZZ^2}\left(\Xi^{(s)}(Z,t)-\sum_{C\in\mathcal{C}_1}\id_{Z\subset C}\Xi^{(s)}(Z,t)\right)
	\end{equation}
	is bounded. Following the arguments in equation (5.22) to (5.24) in \cite{Ogata2d} we still get that
	\begin{equation}
		\sum_{\substack{Z\subset\ZZ^2,\\\nexists C\in\mathcal{C}_1:Z\subset C}}\sup_{t\in[0,1]}\norm{\Xi^{(1)}(Z,t)}\leq \frac{8}{C_F}(e^{2I_F(\Psi)}-1)\sum_{\substack{C_1,C_2\in\mathcal{C}_0,\\ C_1\neq C_2}}M(C_1,C_2)
	\end{equation}
	where
	\begin{equation}
		M(C_1,C_2)\defeq \sum_{m\geq 0}\sum_{\substack{X:\\\forall C\in\mathcal{C}_1,X\cap((C^c)(m))\neq\emptyset,\\X\cap C_1\neq\emptyset,X\cap C_2\neq\emptyset}}\left(\sup_{t\in[0,1]}(\norm{\Psi^{(1)}(X;t)})\abs{X}G_F(m)\right)
	\end{equation}
	is now defined using the new $\mathcal{C}_1$. To bound these $M(C_1,C_2)$ we will (just like in \cite{Ogata2d}) differentiate between two cases. That is, the case where $C_1$ and $C_2$ are adjacent\footnote{We say that $C_1$ and $C_2$ are adjacent if and only if $\#(C_1(1)\cap C_2(1))=\infty$.} and the case where they are not. We begin with the latter. In this case, we still have in complete analogy with the proof in \cite{Ogata2d} that
	\begin{align}
		M(C_1,C_2)\leq b_0(C_1,C_2) &\defeq \sum_{m\geq 0}\sum_{\substack{X:X\cap C_1\neq\emptyset,\\X\cap C_2\neq\emptyset}}\left(\sup_{t\in[0,1]}(\norm{\Psi^{(1)}(X;t)})\abs{X}G_F(m)\right)\\
		&\leq \norm{\Psi_1}_F\sum_{\substack{x\in C_1\\y\in C_2}}F(d(x,y))\sum_{m=0}^\infty G_F(m)<\infty.
	\end{align}
	\footnote{Here $\Psi_1$ is defined through $\Psi_1(X)=\abs{X}^1 \Psi(X)$.} What is now left to show is the case where $C_1$ and $C_2$ are adjacent. Take $\tilde{C}\in\mathcal{C}_1$ such that $C_1\cap\tilde{C}\neq\emptyset$ and $C_2\cap\tilde{C}\neq\emptyset$. Take $L_1=\partial\tilde{C}/C_2$\footnote{Here $\partial \tilde{C}$ means the boundary of $\tilde C$.} and $L_2=\partial\tilde{C}/C_1$. By following the same reasoning that led to equation (5.36) in \cite{Ogata2d} we get that
	\begin{equation}
		M(C_1,C_2)\leq b_0(C_1,C_2/\tilde{C})+b_0(C_1/\tilde{C},C_2)+b_0(\tilde{C}\cap C_2,(C_1\cup C_2)^c)+b_1(C_1\cap\tilde{C},C_2\cap\tilde{C},L_1,L_2)
	\end{equation}
	where
	\begin{align}
		b_1(C_1\cap\tilde{C},C_2\cap\tilde{C},L_1,L_2)&\defeq\sum_{m=0}^\infty \sum_{\substack{X\subset \tilde{C}:\\ X\cap C_1\cap\tilde{C}\neq\emptyset\\X\cap C_2\cap\tilde{C}\neq\emptyset\\ X\cap (\tilde{C}^c(m))\neq\emptyset}}\sup_{t\in[0,1]}\norm{\Psi(X;t)}\abs{X}G_F(m)\\
		&\leq \norm{\Psi_1}_F\sum_{m=0}^{\infty}G_F(m)\left(\sum_{\substack{x\in C_2\cap\tilde{C},\\y\in L_1(m)}}+\sum_{\substack{x\in C_1\cap\tilde{C},\\y\in L_2(m)}}\right)F(d(x,y))<\infty.
	\end{align}
	Since the remainder of the proof can remain unchanged from \cite{Ogata2d}, this concludes the proof of item 3. The proof of item 5 just follows from the fact that for any $\alpha_1\in\textrm{SQAut}(\AA)$ and for any $\alpha_2\in\textrm{HAut}(\AA)$ we have that $\alpha_2\circ\alpha_1\in\textrm{SQAut}(\AA)$. We now only need to comment on item 2. We must show that
	\begin{equation}
		(\gamma^H_{s;t}\circ\gamma^{H_L}_{t;s}\otimes\gamma^{H_R}_{t;s})^{-1}=\gamma^{H_L}_{s;t}\otimes\gamma^{H_R}_{s;t}\circ\gamma^H_{t;s}\in\textrm{GVAut}(\AA).
	\end{equation}
	The proof of this starts analogously to the proof of items 1 and 3. We take $\Psi=H$, $\Psi^{(0)}=H_{\tau(C_\theta^c\cap U)}+H_{\tau^{-1}(C_\theta^c\cap D)}$ and $\Psi^{(1)}=\Psi-\Psi^{(0)}$. Define $\Xi^{(s)}(Z;t)$ again through equation \eqref{eq:PropertiesLocallyGeneratedAutomorphismsProofDefinitionXi}. In analogy to what was done in equation (5.54) in \cite{Ogata2d} we obtain
	\begin{equation}
		\sum_{\substack{Z:Z\nsubseteq \tau(C_\theta^c\cap U)\\\text{and }Z\nsubseteq \tau^{-1}(C_\theta^c\cap D)}}\sup_{t\in[0,1]}\norm{\Xi^{(1)}(Z,t)}\leq \frac{8}{C_F}(e^{2I_F(\Psi)}-1)\sum_{m=0}^\infty\sum_{\substack{X:X(m)\nsubseteq \tau(C_\theta^c\cap U)\\\text{and }X(m)\nsubseteq \tau^{-1}(C_\theta^c\cap D)}} \sup_{t\in[0,1]}\norm{\Psi^{(1)}(X;t)}\abs{X}G_F(m).
	\end{equation}
	If $X$ in the last line has a non-zero contribution, then at least one of the following occurs:
	\begin{enumerate}
		\item $X\cap (W(C_\theta)\cap L)\neq\emptyset$ and $X\cap R\neq\emptyset$.
		\item $X\cap (W(C_\theta)\cap R)\neq\emptyset$ and $X\cap L\neq\emptyset$.
		\item $X\subset W(C_\theta)^c$ and
		\begin{enumerate}
			\item $X\subset U,X\subset D$, or
			\item $X\subset U,X\subset L,X\subset R$ and $X(m)\cap (\tau^{-1}(C_\theta^c\cap D))^c\neq \emptyset$, or
			\item $X\subset D,X\subset L,X\subset R$ and $X(m)\cap (\tau(C_\theta^c\cap U))^c\neq \emptyset$.
		\end{enumerate}
	\end{enumerate}
	This shows that we have a bound
	\begin{align}
		&\sum_{\substack{Z:Z\nsubseteq \tau(C_\theta^c\cap U)\\\text{and }Z\nsubseteq \tau^{-1}(C_\theta^c\cap D)}}\sup_{t\in[0,1]}\norm{\Xi^{(1)}(Z,t)}\\
		\nonumber
		&\leq \frac{8}{C_F}(e^{2I_F(\Psi)}-1)(b_0(W(C_\theta)\cap L,R)+b_0(L,W(C_\theta)\cap R)+b_0(W(C_\theta)^c\cap U,W(C_\theta)^c\cap D)\\
		\nonumber
		&\qquad+b_1(W(C_\theta)^c\cap U\cap L,W(C_\theta)^c\cap U\cap L,L\cap\partial(W(C_\theta)^c\cap U),R\cap\partial(W(C_\theta)^c\cap U))\\
		&\qquad b_1(W(C_\theta)^c\cap D\cap L,W(C_\theta)^c\cap D\cap R,L\cap\partial(W(C_\theta)^c\cap D),R\cap\partial(W(C_\theta)^c\cap D)))<\infty.
	\end{align}
	This concludes the proof.
\end{proof}
This implies certain things for our locally generated automorphisms. From these four statements we can prove the following results:
\begin{lemma}\label{lem:TwoAngleLemmaPart1}
	Take $H$ an interaction such that there exists a $0<\phi<1$ satisfying that $\norm{H}_{f_\phi}\leq 1$. Take $\theta_1$ and $\theta_2$ such that $0<\theta_1<\theta_2<\pi/2$ then for all $\Theta\in\Aut{\AA_{W(C_{\theta_2})^c}}$ and $s,t\in\RR$ there exists an $a_1\in\UU(\AA)$ and a $\tilde{\Theta}\in \Aut{\AA_{W(C_{\theta_1})^c}}$ such that
	\begin{equation}\label{eq:TwoAngleLemmaPart1Equation1}
		\gamma^{H}_{t;s}\circ\Theta\circ\gamma^{H}_{s;t}=\Ad{a_1}\circ\tilde{\Theta}.
	\end{equation}
\end{lemma}
\begin{proof}
	We have that
	\begin{equation}
		\gamma^{H}_{t;s}\circ\Theta\circ\gamma^{H}_{s;t}=\gamma^{H_D}_{t;s}\otimes\gamma^{H_U}_{t;s}\circ\gamma^{H_D}_{s;t}\otimes\gamma^{H_U}_{s;t}\circ\gamma^{H}_{t;s}\circ\Theta\circ\gamma^{H}_{s;t}\circ\gamma^{H_D}_{t;s}\otimes\gamma^{H_U}_{t;s}\circ\gamma^{H_D}_{s;t}\otimes\gamma^{H_U}_{s;t}.
	\end{equation}
	Using that $\gamma^{H}_{s;t}\circ\gamma^{H_D}_{t;s}\otimes\gamma^{H_U}_{t;s}\in\textrm{HAut}_1(\AA)$ we get that there exists some $a\in\AA$ and $\eta\in\Aut{\AA_{C_\theta}}$ such that
	\begin{align}
		\gamma^{H}_{t;s}\circ\Theta\circ\gamma^{H}_{s;t}&=\Ad{a}\circ \gamma^{H_D}_{t;s}\otimes\gamma^{H_U}_{t;s}\circ\eta_{s;t}^{-1}\circ\Theta\circ\eta_{s;t}\circ\gamma^{H_D}_{s;t}\otimes\gamma^{H_U}_{s;t}\\
		&=\Ad{a}\circ \gamma^{H_D}_{t;s}\otimes\gamma^{H_U}_{t;s}\circ\Theta\circ\gamma^{H_D}_{s;t}\otimes\gamma^{H_U}_{s;t}.
	\end{align}
	Since by \ref{lem:PropertiesLocallyGeneratedAutomorphisms} part 3 $\gamma^{H_D}_{s;t}\otimes\gamma^{H_U}_{s;t}\in\textrm{GSQAut}_1(\AA)$ the result follows.
\end{proof}
\begin{lemma}\label{lem:TwoAngleLemmaPart2}
	Take $H$ an interaction such that there exists a $0<\phi<1$ satisfying that $\norm{H}_{f_\phi}\leq 1$. Take $\theta_1$ and $\theta_2$ such that $0<\theta_1<\theta_2<\pi/2$. Then for all $\eta_{g}^{\sigma}\in\Aut{\AA_{\nu^{\sigma}(C_{\theta_1}\cap\sigma)}}$ (where $\sigma\in\{L,R\}$ and $g\in G$) and $s,t\in\RR$ there exist $a_{2},\in\UU(\AA),a_{3,\sigma}\in\UU(\AA_{\nu^{\sigma}(\sigma)})$ and some $\tilde{\eta}_{\sigma}^g\in \Aut{\nu^{\sigma}(\AA_{C_{\theta_2}\cap\sigma)}}$ such that
	\begin{align}
		\label{eq:TwoAngleLemmaPart2Equation1}
		\gamma^{H}_{t;s}\circ\eta_{g}^L\otimes\eta_{g}^R\circ\gamma^{H}_{s;t}&=\Ad{a_2}\circ(\tilde\eta_{g}^L \otimes\tilde\eta_{g}^R)\\
		\label{eq:TwoAngleLemmaPart2Equation2}
		\gamma^{H_\sigma}_{t;s}\circ\eta_g^\sigma\circ\gamma^{H_\sigma}_{s;t}&=\Ad{a_{3,\sigma}}\circ \tilde\eta_{g}^\sigma.
	\end{align}
\end{lemma}
\begin{proof}
	In this proof, take $\Hsplit=H_{\nu^{-1}(L)}+H_{\nu(R)}$. First, we show that equation \eqref{eq:TwoAngleLemmaPart2Equation2} implies equation \eqref{eq:TwoAngleLemmaPart2Equation1}. This is because using equation \eqref{eq:TwoAngleLemmaPart2Equation1} we get that
	\begin{align}
		\gamma^H_{t;s}\circ\eta_g\circ\gamma^{H}_{s;t}&=\gamma^H_{t;s}\circ\gamma^{\Hsplit}_{s;t}\circ\gamma^{\Hsplit}_{t;s}\circ\eta_g\circ\gamma^{\Hsplit}_{s;t}\circ\gamma^{\Hsplit}_{t;s}\circ\gamma^H_{s;t}\\
		&=\gamma^H_{t;s}\circ\gamma^{\Hsplit}_{s;t}\circ\Ad{a_{3,L}\otimes a_{3,R}}\circ\tilde{\eta}_g\circ\gamma^{\Hsplit}_{t;s}\circ\gamma^H_{s;t}.
	\end{align}
	If one now uses the fact that $\gamma^H_{t;s}\circ\gamma^{\Hsplit}_{s;t}\in\textrm{GVAut}_1(\AA)$ (see lemma \ref{lem:PropertiesLocallyGeneratedAutomorphisms} item 2) the implication follows. To finish the proof we now only have to use the fact that the $\gamma^{H_\sigma}_{0;1}$ are in $\textrm{SQAut}_1(\AA_\sigma)$.
\end{proof}
Additionally, when we add the group action, we get:
\begin{lemma}\label{lem:TwoAngleLemmaPart3}
	Take $H$ a $G$-invariant interaction such that there exists a $0<\phi<1$ satisfying that $\norm{H}_{f_\phi}\leq 1$. Take $\theta_1$ and $\theta_2$ such that $0<\theta_1<\theta_2<\pi/2$. Then for all $\eta_{g}^{\sigma}\in\Aut{\AA_{\nu^{\sigma}(C_{\theta_1}\cap\sigma)}}$ (where $\sigma\in\{L,R\}$ and $g\in G$) and $s,t\in\RR$ there exist $a_{2},\in\UU(\AA),a_{3,\sigma}\in\UU(\AA_{\nu^{\sigma}(\sigma)})$ and some $\tilde{\eta}_{\sigma}^g\in \Aut{\nu^{\sigma}(\AA_{C_{\theta_2}\cap\sigma)}}$ such that
	\begin{align}
		\label{eq:TwoAngleLemmaPart3Equation1}
		\gamma^{H}_{t;s}\circ\eta_{g}^L\otimes\eta_{g}^R\circ\beta_g^U\circ\gamma^{H}_{s;t}&=\Ad{a_2}\circ(\tilde\eta_{g}^L \otimes\tilde\eta_{g}^R)\circ\beta_g^U\\
		\label{eq:TwoAngleLemmaPart3Equation2}
		\gamma^{H_\sigma}_{t;s}\circ\eta_g^\sigma\circ\beta_g^{\sigma U}\circ\gamma^{H_\sigma}_{s;t}&=\Ad{a_{3,\sigma}}\circ \tilde\eta_{g}^\sigma\circ\beta_g^{\sigma U}.
	\end{align}
\end{lemma}
\begin{proof}
	In this proof, take again $\Hsplit=H_{\nu^{-1}(L)}+H_{\nu(R)}$. First, we show that equation \eqref{eq:TwoAngleLemmaPart3Equation2} implies equation \eqref{eq:TwoAngleLemmaPart3Equation1}. This is because using (the inverse of) equation \eqref{eq:TwoAngleLemmaPart3Equation1} we get that
	\begin{align}
		&\gamma^H_{t;s}\circ\eta_g\circ\beta_g^U\circ\gamma^H_{s;t}\circ\beta_{g^{-1}}^U\circ(\tilde{\eta}_g)^{-1}\\
		&=(\text{Inner})\circ \gamma^H_{t;s}\circ\eta_g\circ\beta_g^U\circ\gamma^H_{s;t}\circ\gamma^\Hsplit_{t;s}\circ\beta_{g^{-1}}^U\circ(\eta_g)^{-1}\circ\gamma^\Hsplit_{s;t}\circ\underline{\gamma^H_{t;s}\circ\gamma^H_{s;t}}.
	\end{align}
	Now using the fact that by lemma \ref{lem:PropertiesLocallyGeneratedAutomorphisms} item 2, $\gamma^H_{s;t}\circ\gamma^\Hsplit_{t;s}\in\textrm{GVAut}_1(\AA)$ this gives us
	\begin{align}
		&=(\text{Inner})\circ\stkout{\gamma^H_{t;s}\circ\eta_g\circ\beta_g^U\circ\beta_{g^{-1}}^U\circ(\eta_g)^{-1}\circ\gamma^H_{s;t}}.
	\end{align}
	By taking $a_2$ such that $\Ad{a_2}$ is this inner automorphism, we can conclude the proof that equation \eqref{eq:TwoAngleLemmaPart3Equation2} implies equation \eqref{eq:TwoAngleLemmaPart3Equation1}. Now we only have to prove equation \eqref{eq:TwoAngleLemmaPart3Equation2}. To finish the proof we now only have to use the fact that the $\gamma^{H_{\nu^{\sigma}(\sigma)}}_{0;1}$ are in $\textrm{GSQAut}_1(\AA_\sigma)$.
\end{proof}
\begin{lemma}\label{lem:SplittedAutomorphismAfterTranslatedIsVertical}
	There exist maps
	\begin{align}
		A_L&:[0,1]\rightarrow \UU(\AA_L):\lambda\mapsto A_{L}(\lambda)&A_R&:[0,1]\rightarrow \UU(\AA_R):\lambda\mapsto A_{R}(\lambda)
	\end{align}
	both continuous in norm topology and
	\begin{align}
		\Phi^{\mu\nu}:[0,1]\rightarrow\Aut{\AA_{W(C_\theta)^c\cap\mu\cap\nu}}:\lambda\mapsto \Phi^{\mu\nu}(\lambda)
	\end{align}
	where $\mu\in\{U,D\}$ and $\nu\in\{L,\nu(R)\}$, all four continuous in  strong\footnote{Meaning that $\Phi^{\mu\nu}(\lambda)(A)$ is continuous for all $A\in\AA$.} topology, satisfying that
	\begin{align}
		\nu\circ\gamma^{\HsplitTilde}_{0;\lambda}\circ\nu^{-1}\circ\gamma^{\HsplitTilde}_{\lambda;0}&=\gamma^{H_L}_{0;\lambda}\otimes\gamma^{H_{\nu\circ\nu(R)}}_{0;\lambda}\circ\gamma^{\HsplitTilde}_{\lambda;0}\\
		\label{eq:TranslatingSplittedTimeEvolutionAppendix}
		&=\Ad{A_L(\lambda)}\otimes\Ad{A_R(\lambda)}\circ\bigotimes_{\substack{\mu\in\{U,D\},\\\nu\in\{L,\nu(R)\}}}\Phi^{\mu\nu}(\lambda)
	\end{align}
	and that $\Phi^{\mu\nu}\circ\beta_g=\beta_g\circ\Phi^{\mu\nu}$.
\end{lemma}
\begin{proof}
	The proof of this is very similar to all the other proofs in this section. We will go over the highlights. We want to define an $A_L(\lambda)$ and a $\Phi^{U,L}\otimes\Phi^{D,L}$ satisfying
	\begin{equation}
		\Ad{A_L(\lambda)}\circ\Phi^{U,L}\otimes\Phi^{D,L}(\lambda)=\gamma^{H_L}_{0;\lambda}\circ\gamma^{H_{\nu^{-1}(L)}}_{\lambda;0}
	\end{equation}
	and similarly for the right side. We will only work out this side as the other calculation is analogous. We do this in two steps. First, we find an interaction $\Xi$ satisfying
	\begin{equation}
		\gamma^\Xi_{0;\lambda}=\gamma^{H_L}_{0;\lambda}\circ\gamma^{H_{\nu^{-1}(L)}}_{\lambda;0}.
	\end{equation}
	We can do this by taking the usual
	\begin{equation}
		\Xi(Z,t)\defeq \sum_{m=0}^{\infty}\sum_{\substack{X\subset Z\\X(m)=Z}}\Delta_{X(m)}(\gamma^{H_{\nu^{-1}(L)}}_{0;t}(\tilde{H}(X,t)))
	\end{equation}
	where $\tilde{H}\defeq H_L-H_{\nu^{-1}(L)}$. We can now define
	\begin{equation}
		\Phi^{\mu L}(\lambda)\defeq\gamma^{\Xi_{W(C_\theta)^c\cap\mu\cap L}}_{0;\lambda}.
	\end{equation}
	This means that we can satisfy \ref{eq:TranslatingSplittedTimeEvolutionAppendix} if
	\begin{align}
		\Ad{A_L}&=\gamma^{\Xi}_{0;\lambda}\circ\gamma^{\Xi'}_{\lambda;0}&\Xi'&\defeq \Xi_{W(C_\theta)^c\cap U\cap L}+\Xi_{W(C_\theta)^c\cap D\cap L}.
	\end{align}
	Using the same argument as before, this can be realized when we have
	\begin{equation}
		\Ad{A_L(\lambda)}=\gamma^{\tilde\Xi}_{0;\lambda}
	\end{equation}
	where
	\begin{equation}
		\tilde\Xi(Z,t)\defeq \sum_{m=0}^{\infty}\sum_{\substack{X\subset Z\\X(m)=Z}}\Delta_{X(m)}(\gamma^{\Xi'}_{0;t}(\Xi(X,t)-\Xi'(X,t))).
	\end{equation}
	The last part of the proof is to show that $\norm{\sum_{Z\in\mathfrak{B}_{\ZZ^2}}\tilde\Xi(Z,t)}<\infty$ which can be done using similar techniques as before. This means that we can simply define
	\begin{equation}
		A_L(\lambda)\defeq\mathcal{T}\exp(-i\int_0^\lambda \d s\: \sum_{Z\in \mathfrak{B}_{\ZZ^2}}\tilde{\Xi}(Z,s))
	\end{equation}
	and this is by construction bounded and continuous in $\lambda$.
\end{proof}


%%%%%%%%%%%%%%%%%%%%%%%%%%%%%%%%%%%%%%%%%%%%%%%%%%
% Keep the following \cleardoublepage at the end of this file, 
% otherwise \includeonly includes empty pages.
\cleardoublepage

% vim: tw=70 nocindent expandtab foldmethod=marker foldmarker={{{}{,}{}}}
