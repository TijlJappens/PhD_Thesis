% !TeX root = ../../thesis.tex
\chapter{Conclusion}\label{ch:conclusion}

We have shown how the mathematical tools in section \ref{ch:Tools_For_Quantum_many_Body} can be used to produce multiple SPT indices and that these indices are robust and in some cases complete.
\\\\
In chapter \ref{ch:LoopSPT} we had a quasi-local $C^*$-algebra over $\ZZ$. We defined the concept of $G$-loops in definition \ref{def: G-loop} and defined the concept of stable $G$-homotopy in definition \ref{def: stable homotopy}. We then produced an $H^1(G,\TT)$-valued index for $G$-loops which we called the charge pumping index. We then showed that two $G$-loops are stably homotopic if and only if they had the same index. We denoted this result in theorem \ref{thm: classification loops}.
\\\\
In chapter \ref{ch:TranslationSPT} we had a quasi-local $C^*$-algebra over $\ZZ^2$. We defined the concept of SRE in definition \ref{def:sre} and defined an on-site symmetry and two translation symmetries. We defined the sets $S_0,S_1,S_2$ and $S_3$ as the subsets of SRE states that satisfied some symmetry. We also defined some equivalence relation on these sets (see definition \ref{def:S_i_WithEquivelenceRelation}). We then defined an $H^2(G,\TT)$-valued index for every translation symmetry and an $H^1(G,\TT)$-valued index in the case where there are two translation symmetries. Finally, we showed that the index is invariant under the choice of representative in the stable equivalence relation. This we formulated in lemma \ref{lem:IndicesConstantOnStableEquivalenceClasses}.
\\\\
There is an endless amount of further research that can be conducted that extends the ideas in this work. I give some ideas here:
\begin{itemize}
	\item We could extend the loop classification to other dimensions. More specifically, in 3d it might be possible that the boundary state of a cyclic process is a state with non-trivial invertible topological order.
	\item It could also be interesting to extend the loop classification to states that have non-trivial anyonic topological order. In this case as well, there are loops that might be non-trivial even without the application of any symmetry. In fact, in \cite{Aasen_2022} it has been shown there are locally generated loops of the toric code ground-state that exchange the e and m excitations. Such situations could really benefit from a rigorous treatise.
	\item There is another interesting generalization of the loop classification coming from Kitaev's $\Omega$ spectrum hypothesis (see e.g. \cite{Xiong_2018}). The generalization is that instead of comparing the loops of $d+1$ dimensional SPTs with their $d$-dimensional boundary states, we compare maps from the suspension of some differential manifold $\Sigma M$ to $d+1$-dimensional SPTs to maps from $M$ to the $d$-dimensional SPTs. This is related to the higher Berry curvature index you will find in \cite{Kapustin_2022} and \cite{artymowicz2023quantization}.
\end{itemize}
In conclusion, multiple directions hold the promise of revealing or analysing fascinating new phenomena that could enrich our understanding of quantum matter.
%%%%%%%%%%%%%%%%%%%%%%%%%%%%%%%%%%%%%%%%%%%%%%%%%%
% Keep the following \cleardoublepage at the end of this file, 
% otherwise \includeonly includes empty pages.
\cleardoublepage

% vim: tw=70 nocindent expandtab foldmethod=marker foldmarker={{{}{,}{}}}
