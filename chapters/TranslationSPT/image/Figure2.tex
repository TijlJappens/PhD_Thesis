\documentclass[preview=false]{standalone}
\begin{document}

\begin{figure}
	\centering
	\def\s{0.5}
	\resizebox{0.45\textwidth}{!}{%
		\begin{tikzpicture}
			\fill[fill=red!30!white] (0,0) -- (2.5*\s,4*\s) -- (4*\s,4*\s) -- (4*\s,2.5*\s);
			\fill[fill=red!30!white] (0,0) -- (2.5*\s,-4*\s) -- (4*\s,-4*\s) -- (4*\s,-2.5*\s);
			\fill[fill=red!30!white] (0,0) -- (-2.5*\s,4*\s) -- (-4*\s,4*\s) -- (-4*\s,2.5*\s);
			\fill[fill=red!30!white] (0,0) -- (-2.5*\s,-4*\s) -- (-4*\s,-4*\s) -- (-4*\s,-2.5*\s);
			
			\draw[draw=black,line width=0.3mm] (-4*\s,4*\s) -- (4*\s,-4*\s);
			\draw[draw=black,line width=0.3mm] (-4*\s,-4*\s) -- (4*\s,4*\s);
			
			\node at (2*\s,0) {$1$};
			\node at (0,2*\s) {$2$};
			\node at (-2*\s,0) {$3$};
			\node at (0,-2*\s) {$4$};
		\end{tikzpicture}}
	\caption{This figure indicates four different areas of $\ZZ^2$ and we will take $\pi_0=\bigotimes_{i=1}^4\pi_{0,i}$ where each of the $\pi_{0,i}$ is an irreducible representation of the restriction of $\AA$ to each of the areas. The red areas are chosen such that they do not overlap with the support of $\tilde{\eta}_g$ and $\eta_g$.}
	\label{fig:IndexInvariantUnderRotationDecomposition}
\end{figure}
\end{document}